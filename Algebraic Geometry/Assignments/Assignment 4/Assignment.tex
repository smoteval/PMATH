\documentclass[12pt]{article}

\usepackage{fullpage,url,amssymb,epsfig,color,xspace,enumerate}
\usepackage{amsmath}

\renewcommand{\thesubsection}{Problem \arabic{subsection}}

\begin{document}
\begin{center}
  {\Large\bf University of Waterloo}\\
  \vspace{3mm}
         {\Large\bf Algebraic Geometry - Summer 2015}\\
         \vspace{2mm}
                {\Large\bf Assignment 4}\\
                \vspace{3mm}
                \textbf{Sina Motevalli 20455091}
\end{center}
\section*{Problem 1}
Let $f_1,f_2,...,f_r\in k[x_1,...,x_n]$ such that $I=<f_1,...,f_r>$. 
For each $i \in \{1,2,..,r\}$ let $f_i=\prod_{j} f_{ij}$ be a factorization of $f_i$ into irreducibles. \\
So $V(f_i)=\cup_j V(f_{ij})$ is a decomposition of $V(f_I)$ into irreducible components. \\
$V(I)=\cup_{(j_1,...,j_r)} (V(f_{1j_1}) \cap ... \cap V(f_{rj_r}))$ is a decomposition of $V(I)$ into irreducibles. \\
Fix $(j_1,...,j_r)$. We have that \\
$\dim ( V(f_{1j_1}) = n-1$\\ 
$\dim (V(f{1j_1} \cap V(f_{2j_2})) \ge (n-1)-1=n-2$ \\
$\dim (V(f{1j_1} \cap V(f_{2j_2}) \cap V(f_{3j_3})) \ge (n-2)-3=n-3$
... \\ 
$\dim (V(f_{1j_1}) \cap ... \cap V(f_{rj_r})) \ge n-r$. \\
Thus every irreducible component of $V(I)$ has dimension $\ge n-r$.

\section*{Problem 2}
Let $\varphi:X \rightarrow Y$ be a dominant rational map. \\
Since $\varphi$ is dominant, the pull-back $\varphi^* :k(Y) \rightarrow k(X)$ is well-defined. \\
Let $f_1,...,f_s \in k(Y)$ be algebraically independent.  \\
Let $h_i = \varphi^*(f_i)=f_i \circ \varphi$ for all $i \in \{1,2,...,s\}$. \\
Assume for a contradiction that $h_1,...,h_s \in k[X]$ is algebraically dependent. \\
So there exist $g \in k[x_1,...,x_s]$ such that
$g(h_1,...,h_s)=0$. We have:
\begin{eqnarray*}
g(h_1,...,h_s)=0 &\Rightarrow &
g(f_1 \circ \varphi,...,f_s \circ \varphi) =0 \\ &\Rightarrow &
g(f_1,...,f_s) \circ \varphi =0 \\ &\Rightarrow &
g(f_1,...,f_2) =0
\end{eqnarray*}
This contradicts with $f_1,...,f_s \in k(Y)$ being algebraically independent.\\
Thus
$h_1,...,h_s \in k[X]$ are algebraically independent. \\
Hence $\dim Y \le \dim X$.



\clearpage
\section*{Problem 3}
\subsection*{Part a}
Let $I$ be a prime ideal of $\Gamma(X)$ contained in $M_p$ and let $J$ be a prime ideal of $O_p(X)$.The correspondence is given by
\begin{eqnarray*}
I &\rightarrow & IO_p(X) \\
J &\rightarrow & J \cap \Gamma(X)
\end{eqnarray*}
To prove this is one-to-one we need to argue that $IO_p(X) \cap \Gamma(X)=I$. \\
Note that since $IO_p(X)$ is the prime ideal in $O_p(X)$ generated by $I$ and every polynomial in $I$ evalutes to $0$ at $p$, if $a,b \in \Gamma(X)$ and $f \in I$ then
$f\frac{a}{b}$ is a polynomial iff $b(p)=f(p)$ but since $f(p)=0$ this implies that
$\frac{a}{b} \not\in O_p(X)$. So every polynomial in $IO_p(X)$ is in $I$. \\
Hence $IO_p(X) \cap \Gamma(X)=I$ and we are done.
\subsection*{Part b}
It's easy to see that there is a one-to-one correspondence between subvarieties of $X$ containing $p$ and prime ideals in $\Gamma(X)$ contained in $M_p$ because every subvariety of $X$ containing $p$ corresponds to a prime ideal in $\Gamma(X)$ contained in $M_p$ and every prime ideal in $\Gamma(X)$ contained in $M_p$ corresponds to a subvariety of $X$. So by part (a), there is a one-to-one correspondence between the prime ideals of $O_p(X)$ and subvareties of $X$ containing $p$.
\subsection*{Part c}
Let $P_0 \subset P_1 \subset ... \subset P_n=O_p(X)$ be the longest chain of prime ideals in $O_p(X)$. By part (b), there is a one-to-one correspondence between $P_i$'s and the subvarieties of $X$ containing $p$. So this chain corresponds to the longest chain of irreducible closed subsets of $X$ containing $p$ (Because if there is a longer chain of irreducible closed subsets of $X$ containing $p$, then by the one-to-one correspondence we can get a longer  chain of prime ideals in $O_p(X)$). Thus $\dim X=\dim O_p(X)$. \\



\clearpage
\section*{Problem 4}
\subsection*{Part a}
Let $\varphi^*: O_q(Y) \rightarrow O_p(X)$ be the extension of pull-back that sends 
$f \in O_q(Y)$ to $f \circ \varphi$. Note that if $f \in O_q(Y)$, then $f$ is defined on $q$
and since $q=\varphi(p)$, $(f \circ \varphi)(p)=f(\varphi(p))=f(q)$ is defined therefore
$f \circ \varphi \in O_p(X)$. So the pull-back is well-defined \\
Let $f \in M_q$. Then $(f \circ \varphi)(p)=f(\varphi(p))=f(q)=0$, thus $f \circ \varphi \in M_p$.
\\
So $\varphi^*(M_q(Y)) \subset M_p(X)$. \\
$\varphi^*$ cannot be extended to all of $k(Y)$ because some elements of $k(Y)$ are not defined on all of $Y$. \\
Now assume $\varphi$ is an isomorphism. Let $f \in M_p(X)$.
There exist $a,b \in \Gamma(X)$ with $a(p)=0$ and $b(p) \not= 0$ and $f = \frac{a}{b}$. \\
Since $\varphi^*$ is an isomorphism, $\Gamma(X)$ and $\Gamma(Y)$ are isomorphis. So there exists $a',b' \in \Gamma(Y)$ such that $a=a' \circ \varphi$ and $b=b' \circ \varphi$. We have: \\
$a'(q)=a'(\varphi(p))=a(p) =0$ and $b'(q)=b'(\varphi(p))=b(p) \not=0$ so
$g=\frac{a}{b} \in M_q(Y)$. \\
Now note that $\varphi^* (g) = \frac{a'}{b'} \circ \varphi = \frac{a}{b}=f$. Thus $\varphi^*(M_q(Y))=M_p(X)$.

\subsection*{Part b}
Let $\varphi:X \rightarrow Y$ be an isomorphism. Let $p \in X$. 
By part (a), $O_p(X)$ is isomorphic to $O_{\varphi(p)}(Y)$. \\
X is smooth at $p$ if and only if $O_p(X)$ is regular which happens if and only if 
$O_{\varphi(p)}(Y)$ is regular because $O_p(X)$ is isomorphic to $O_{\varphi(p)}(Y)$ and
$O_{\varphi(p)}(Y)$ is regular if and only if Y is smooth at $\varphi(p)$. 



\subsection*{Part c}
No. We will show that X is not smooth at $(0,0,0)$ but Y is.Therefore by part (b), $X$ and $Y$ are not isomorphic. \\
First note that $\dim X= \dim Y=2$ because $x^2+y^2-z^2$ and $x^2+y^2-z$ are irreducible polynomials. \\
We have $Jac(x^2+y^2-z^2) (0,0,0) = (2x,2y,-2z) (0,0,0)=(0,0,0)$.
So $\dim (T_.0(X))=3$ but $\dim(X)=2$ thus $\dim (T_0(X)) \not= \dim(X)$. \\
We also have $jac(x^2+y^2-z)(0,0,0)=(2x,2y,-1)(0,0,0)=(0,0,-1)$. 
So $\dim (T_.0(Y))=2$ and $\dim(Y)=2$. Thus $\dim(Y)=\dim (T_0(Y))$.



\clearpage
\section*{Problem 5}
\subsection*{Part a}
\begin{enumerate}
\item[(i)] Since $k$ is a field, $(k,+)$ is a group and $m(x,y)=x+y$ and $i(x)=-x$ and $e=0$ and it can be identified by $\mathbb{A}^1$. So the additive group is an affine algebraic group.
\item[(ii)] This is a group with $m(x,y)=xy$ and $i(x)=x^{-1}$ and $e=1$. \\
Also since this group can be identified by $V(xy-1)$, the multiplicative group is an affine algebraic group.
\item[(iii)] This is a group where matrix multiplication is given by polynomials in the entries of the matrcies and therefore is a polynomial map. Inversion is also given by polynomials in the entries of a matrix divided by the determinant but with the variable $t_{n^2+1}=\frac{1}{\det A}$, so inversion is also a polynomial map. \\
Also note that $GL_n$ can be identified by $V(\det(A)t_{n^2+1}-1)$ where
$\det(A)t_{n^2+1}-1$ is an irreducible polynomial (because $\det A$ is irreducible and $t_{n^2+1}$ is a variable not used in $\det A$). So the general linear group is an affine algebraic group and $GL_n$ is irreducible.
\item[(iv)] This is a group where matrix multiplication is given by polynomials in the entries of the matrcies and therefore is a polynomial map. Inversion is also given by polynomials in the entries of a matrix divided by the determinant but determinant is $1$, so inversion is also a polynomial map. \\
Also since this group can be identified with $V(\det (A)-1)$, special linear group is an affine algebraic group.
\item[(v)] Orthogonal group is a group where matrix multiplication is given by polynomials in the entries of the matrcies and therefore is a polynomial map. Inversion is also given by polynomials in the entries of a matrix divided by the determinant but determinant is either $1$ or $-1$, so inversion is also a polynomial map. \\
$O_n$ can be identified by $V(AA^T-I,\det A -1) \cup  V(AA^T-I,\det A +1)$,so it is an affine algebraic group. \\
Since $O_n$ and $SL_n$ are both affine algebraic groups, their intersection $SO_n$ is also an affine algebraic group.
\end{enumerate}


\clearpage
\subsection*{Part b}
\begin{enumerate}
\item[(i)] Let $g \in G$. Let $g: G \rightarrow G$ be the map that sends $h$ to $m(g,h)$. \\
Claim 1: $g$ is an isomorphism. \\
Proof of claim 1: $g$ is clearly a polynomial map (because $m$ is a polynomial map) and
$g^{-1}: G \rightarrow G$ which sends $h$ to $m(i(g),h)$ is also a polynomial map because both $m$ and $i$ are polynomail maps. \\
Let $G_1$ be an irreducible component of $G$. \\
Claim 2: $g(G_1)$ is an irreducible component of $G$. \\
Proof of claim 2: Since $g$ is an isomorphism, we know that $g(G_1)$ is a closed subset of $G$. Now assume for a contradiction that $g(G_1)=A_1 \cup A_2$ where $A_1,A_2$ are distinct closed subsets of $G$. Then we have $G_1=g^{-1}(A_1) \cup g^{-1}(A_2)$ is a decomposition of $G_1$ into two distict closed subsets contradicting the irreducubility of $G_1$. \\
Let $G_1$ and $G_2$ be irreducible components of $G$. \\
Assume $g \in G_1 \cap G_2$. \\
Claim 3: $G_1=G_2$. \\
Proof of claim 3:
By claim 2 $g^{-1}G_1$ is irreducible (Consider the map $g^{-1}:G \rightarrow G$ that sends $h$ to $m(i(g),h)$). Also $g^{-1}G_2$ is irreducible.\\
Note that $e =g^{-1}g \in g^{-1}G_1$ and $e =g^{-1}g \in g^{-1}G_2$ since $g  \in G_1 \cap G-2$. So $e \in g^{-1}G_1 \cap g^{-1}G_2$.  \\
Claim 4: $G_1$ is isomorphic to $G^{\circ}$. \\
Proof of claim 4:
Let $g: G \rightarrow G$ be the map that sends $h$ to $m(i(g),h)$. Note that
$g(g)=e$. So $g$ restricted to $G_{i_0}$ is an isomorphism between 
$G_{i_0}$ and $G^{\circ}$ because $g$ sends irreducible components to irreducible components and we know from part (a) that irreducible components are pairwise disjoint.
\\
\item[(ii)]
Let $g: G \rightarrow G$ be the map that sends $h$ to $m(i(g),h)$. Note that
$g(g)=e$. So $g$ restricted to $G_{i_0}$ is an isomorphism between 
$G_{i_0}$ and $G^{\circ}$ because $g$ sends irreducible components to irreducible components and we know from part (a) that irreducible components are pairwise disjoint.
\\
Since there is an isomorphism between $G_{i_0}$ and $G^{\circ}$ that sends $g$ to $e$, we have that $T_eG^{\circ}$ is isomorphic to $T_gG_{i_0}$.
\clearpage
\item[(iii)] We compute the Lie algebras for $n=2$. \\
Note that $SL_2$ is irreducible because $\det A -1$ is an irreducible polynomial. We have: 
\begin{eqnarray*}
T_{I_{2 \times 2}} SL_2&=&
\{
(a,b,c,d): \ jac(\det 
\begin{bmatrix}
x & y \\
z & w
\end{bmatrix}
-1)(I_{2 \times 2}).(a,b,c,d)=0
\}
\\ &=&
\{
(a,b,c,d): \ jac(xw-yz-1)(I_{2 \times 2}).(a,b,c,d)=0
\}
\\ &=&
\{
(a,b,c,d): (w,-z,-y,x)(I_{2 \times 2}).(a,b,c,d)=0
\}
\\ &=&
\{
(a,b,c,d): (0,-1,-1,0).(a,b,c,d)=0
\}
\\ &=&
\{
(a,b,,c,d) : -b-c =0
\} 
\\ &=&
\{
A \in M_{2\times 2}: \ Tr(A)=0
\}
\end{eqnarray*}
Given $A=
\begin{bmatrix}
x & y \\
z & w
\end{bmatrix}$
We have $SO_2=V(xw-yz-1,x^2+y^2,xz+yw-1,z^2+w^2)$. \\
$T_{I_{2 \times 2}}SO_2$ is the set of vectors $(a,b,c,d)$ satisfying 
\begin{eqnarray*}
(w,-z,-y,x)(I_{2 \times 2}).(a,b,c,d) &= &0 \\
(2x,2y,0,0)(I_{2 \times 2}).(a,b,c,d) &=& 0 \\
(z,w,x,y)(I_{2 \times 2}).(a,b,c,d) &=& 0 \\
(0,0,2z,2w)(I_{2 \times 2}).(a,b,c,d) &=& 0
\end{eqnarray*}
So 
\begin{eqnarray*}
-b-c &=& 0\\
2a &=& 0 \\
a+d &=& 0 \\
2d &=& 0
\end{eqnarray*}
So
\begin{eqnarray*}
T_{I_{2 \times 2}}SO_2 &=&
\{ (a,b,c,d): 
b+c=2b=a+d=2c=0
\}
\\ &=&
\{A \in M_{2 \times 2}: A+A^T=0\}
\end{eqnarray*}



\end{enumerate}

\end{document}