\documentclass[12pt]{article}

\usepackage{fullpage,url,amssymb,epsfig,color,xspace,enumerate}
\usepackage{amsmath}

\renewcommand{\thesubsection}{Problem \arabic{subsection}}

\begin{document}
\begin{center}
  {\Large\bf University of Waterloo}\\
  \vspace{3mm}
         {\Large\bf Alebraic Geometry - Summer 2015}\\
         \vspace{2mm}
                {\Large\bf Assignment 1}\\
                \vspace{3mm}
                \textbf{Sina Motevalli 20455091}
\end{center}
\section*{Problem 1}
We assume $L=V(x_2-(a_2x_1-b_2),x_3-(a_3x_1-b_3),...,x_n-(a_nx_1-b_n))$.
Since $L \not\subset X$, $X \not= \mathbb{A}^n$. So there exist a non-zero polynomial $F \in k[x_1,...,x_n]$ such that $X \subset V(F)$. Assume for a contradiction that $X \cap L$ is an infinite set. So $L \cap V(F)$ is an infinite set. We have:
\begin{eqnarray*}
L \cap V(F) &=&
V(x_2-(a_2x_1-b_2),x_3-(a_3x_1-b_3),...,x_n-(a_nx_1-b_n)) \cap V(F)
\\ &=&
V(F(x_1,a_2x_1-b_2,...,a_nx_1-b_n)) \in k[x_1]  \ \ since \ \ \ x_i-(a_ix_1-b_i)=0
\end{eqnarray*}  
So $L \cap V(F)$ is the zero set of some polynomial in one variable, but there can only be finitely many points in such a set. Contradiction.

\clearpage
\section*{Problem 2}
\subsection*{Part a}
I will use problem 1 here. Notice that the pints in $\mathbb{R}^2$ whose polar coordinates satisfy $r=\Theta$ intersect the line $y=0$ at every point $(2k\pi, 0)$ for all $k \in \mathbb{N}$. Thus By problem 1 the set is not algebraic.
\subsection*{Part b}
The set simply represents the circle of radius $\frac{1}{2}$ centered at $(\frac{1}{2},0)$. So it is the zero set of the polynomial $(x-\frac{1}{2})^2+y^2-\frac{1}{4}$. Thus the set is algebraic.

\subsection*{Part c}
Claim: $V(x^2+z^2-1,y-1) =\{(\cos t,1,\sin t):t \in \mathbb{R}\} \subset \mathbb{R}^3$ \\
Proof: We have
$$V(x^2+z^2-1,y-1) = V(x^2+z^2-1) \cap V(y-1)$$
Also, $V(y-1)$ is the set of points with $y=1$ and
$V(x^2+z^2-1)$ is the cylinder of radius 1 around around $y$ axis. So the intersection of $V(y-1)$ and $V(x^2+z^2-1)$ is the circle of radius 1 centered at $(0,1,0)$ which can be paramtrized by $\{(\cos t,1,\sin t):t \in \mathbb{R}\}$. Thus the set is algebraic.
\subsection*{Part d}
Let $L=\{(1,t,0): t \in \mathbb{R}\}$ be the parametrization of a line. It's easy to see that the set in the question intersects $L$ at every point of the form $(1,2k\pi,0)$ for all $k \in \mathbb{Z}$. So by problem 1, the set is not algebraic.
\subsection*{Part e}
This is clearly $V(x^2+y^2+z^2+w^2-1)$ because the set is just $S^3$ (unit 3-shere). Thus it is algebraic.
\subsection*{Part f}
We know that there is a ring homomorphism between $\mathbb{R}^4$ to $\mathbb{C}^2$. This ring homomorphism, by definition, carries zero sets from $\mathbb{R}^4$ to
$\mathbb{C}^2$. We proved in the previous part that $\{v \in \mathbb{R}^4: |v|=1\}$ is the zero set of the polynomial $x^2+y^2+z^2+w^2-1$. But if $z=x+iy$ and $w=z+it$, then $x^2+y^2+z^2+t^2=1 \Rightarrow |z|^2+|w|^2=1$. So this must be an algebraic set.


\clearpage
\section*{Problem 3}
We know from class that the only algebraic sets (Closed sets) on $\mathbb{R}$ are of one the following:
\begin{enumerate}
\item $\emptyset $
\item $\mathbb{R}$
\item Sets with finitely many points
\end{enumerate}
Now let $p,q \in \mathbb{R}$. 
Let $U$ and $V$ be neighbours of $p$ and $q$ respectively. So the complement of each of $U$ and $V$ contains only finitely many points. Just pick a point $r \in \mathbb{R}$ that is not among the finitely many points in $U^{c} \cup V^{c}$. Then $r \in U \cup V$.
Thus $U \cup V \not= \emptyset$.
Hence Zariski topology on $\mathbb{R}$ is not housdorff.
\\
Note that we proved a stronger statement than the zariki topology not being housdorff, namely, that every two non-empty open sets intersect (with respect to zariski topology).

\section*{Problem 4}
We know from class that a point $a \in \mathbb{A}^n$ as a set $\{a\}$ is an algebraic set. This is because given 
$a=(a_1,...,a_n) \in \mathbb{A}^n$, we have 
$V(x_1-a_1,x_2-a_2,...,x_n-a_n)=\{a\}$.
\\
So if $k$ is a finite field, every subset of $\mathbb{A}^n$ is a finite set and is therefore algebraic. This readily implies that every subset of $\mathbb{A}^n(k)$ is both open and closed (because the complement of every set is finite and therefore closed). We also have that the Zariski topology is housdorff here because given two distinct points $p,q \in \mathbb{A}^n$, the sets $\{p\}$ and $\{q\}$ are open sets with no intersection.
\section*{Problem 5}
We just need to argue that the complement of the set of $n \times n$ invertible matrcies is an algebraic set. In other words we need to show that the set of non-invertible matrcies is an alebraic set. This can be easily achieved because non-invertible matrcies are exactly the matrcies with 0 determinent. So the set of non-invertible matrcies is $V(\det_n)$ and thus is an algebraic set.

\clearpage
\section*{Problem 6}
Let $S_1 \subset k[x_1,...x_n]$ and $S_2 \subset k[y_1,..,y_m]$ be finite sets such that $X=V(S_1)$ and $Y=V(S_2)$. (This is possible since $X$ and $Y$ are algebraic sets). \\
For each $f \in k[x_1,...x_n]$ we define a polynomial
$f' \in k[x_1,...,x_n,y_1,...,y_m]$ such that
$$f'(a_1,...,a_n,b_1,...,b_m)=f(a_1,...,a_n)$$ for any 
$b_1,...,b_m \in \mathbb{A}^m$. \\
Similarly 
for each $g \in k[y_1,...y_m]$ we define a polynomial
$g' \in k[x_1,...,x_n,y_1,...,y_m]$ such that
$$g'(a_1,...,a_n,b_1,...,b_m)=g(b_1,...,b_m)$$ for any 
$a_1,...,a_n \in \mathbb{A}^n$. \\
Now let $S_1'$ and $S_2'$ be $S_1$ and $S_2$ with every polynomial in each set extended to a polynomial in $k[x_1,...,x_n,y_1,...,y_m]$ in the manner described above. \\
Claim: $V(S_1 \cup S_2)=V \times W$. \\
Proof: Let $a \in V(S_1' \cup S_2')$. So for any $f' \in S_1', f'(a)=0$, so $f(a)=0$ implying $a \in V(S_1)=V$.
Also for any $g' \in S_2', g'(a)=0$, so $g(a)=0$ implying that $a \in V(S_2)=W$. Hence $a \in V \times W$ proving
$V(S_1 \cup S_2) \subset V \times W$. \\
Let $a=(a_1,...,a_n,b_1,...,b_m) \in V \times W$. Let
$f' \in S_1' \cup S_2'$. We have the following cases:
\begin{enumerate}
\item[($f' \in S_1'$)] 
In this case $f'(a_1,...,a_n,b_1,...,b_m)=f(a_1,...,a_n)=0$
\item[($f' \in S_2'$)] 
In this case $f'(a_1,...,a_n,b_1,...,b_m)=f(b_1,...,b_m)=0$
\end{enumerate}
Thus $a \in V(S_1' \cup S_2')$ proving $
V \times W \subset V(S_1' \cup S_2')$.
 
\clearpage 
\section*{Problem 7}
\subsection*{Part a}
As I mentioned before in this assignment, the closed sets on the affine line are $\mathbb{A}^1$ or $\emptyset$ or a finite set of points. Let $X=\{a_1,...,a_k\}$ and $Y=\{b_1,...,b_m\}$ be closed sets in $\mathbb{A}^1$. 
Then 
$$A \times B = \{(a,b): a \in \{a_1,...,a_k\}, b \in \{b_1,...b_m\} \}$$ 

$$A \times \mathbb{A}^1  = \{(a,b): a \in \{a_1,...,a_k\}, b \in \mathbb{A}^1 \}$$ 

$$ \mathbb{A}^1 \times A  = \{(a,b): a \in \mathbb{A}, b \in \{a_1,...,a_n\} \}$$
So the closed sets in the Zariski topology of $\mathbb{A}^1 \times \mathbb{A}^1$ are
\begin{enumerate}
\item $\emptyset$
\item $\mathbb{A}^1 \times \mathbb{A}^1$
\item Finite sets
\item Finite collection of vertical lines
\item Finite collection of horizontal lines
\end{enumerate}
\subsection*{Part b}
We have that $V(x-y)$ is a closed set in the zariski topology of $\mathbb{A}^2$. But it is not a closed set in the product zariski topology $\mathbb{A}^1 \times \mathbb{A}^1$ as we can see from the previous part of the problem.



\end{document}