\documentclass[12pt]{article}

\usepackage{fullpage,url,amssymb,epsfig,color,xspace,enumerate}
\usepackage{amsmath}

\renewcommand{\thesubsection}{Problem \arabic{subsection}}

\begin{document}
\begin{center}
  {\Large\bf University of Waterloo}\\
  \vspace{3mm}
         {\Large\bf Algebraic Geometry - Summer 2015}\\
         \vspace{2mm}
                {\Large\bf Assignment 3}\\
                \vspace{3mm}
                \textbf{Sina Motevalli 20455091}
\end{center}
\section*{Problem 1}
\subsection*{Part a}
We know from linear algebra that every element of $O(n,k)$ has determinant either $1$ or $-1$. Thus
$O(n,k)=(V(\det_n -1) \cap O(n,k)) \cup (V(\det_n +1)\cap O(n,k))$. So the set is reducible and therefore is not a variety. 
\subsection*{Part b}
We just need to note that
$$SU(2,\mathbb{C})=\{
\begin{bmatrix}
a & -\bar{b} \\
b & \bar{a}
\end{bmatrix}
: a,b \in \mathbb{C}, |a|^2+|b|^2=1
\}
$$
As I argued in Assignment 1, this intersects the line $z=1$ infinitely many times and therefore is not an algebraic set.


\subsection*{Part c}
The polynomial map
$\phi: \mathbb{A}^1 \rightarrow V(xz-y^2,yz-x^3,z^2-x^y)$ that sends $t$ to $(t^3,t^4,t^5)$ is surjective. Therefore since $\mathbb{A}^1$ is irreducible, so is $X=\phi(\mathbb{A}^1)$.



\section*{Problem 2}
It is sufficient to show that $k(\mathbb{A}^1)$ is not isomorphic to $k(V(y^2-x^3))$.\\
We have that $k(V(y^2-x^3))$ is isomorphic to $k[t^2,t^3]$ and $k(\mathbb{A}^1)$ is isomorphic to $k[t]$. \\
Note that $k[t]$ is a UFD, but $k[t^2,t^3]$ is not because $t^8=t^3t^3t^2=(t^2)^3$ has two factorizations.
Hence $k(\mathbb{A}^1)$ is not isomorphic to $k(V(y^2-x^3))$.


\clearpage
\section*{Problem 3}
\subsection*{Part a}
Note that $\phi^*$ sends $g+I(Y)$ to $g \circ \phi
+I(X)$.\\ 
So $\phi^*$ is injective means
$g \in I(Y) \Longleftrightarrow g \circ \phi \in I(X)$. \\
Since $g \circ \phi \in I(X) \Longleftrightarrow
g \in I(\phi(X))$, we have that
$\phi^{*}$ is injective if and only if
$$I(Y)=I(\phi(X))$$
which is equivalent to image of $\phi$ under $X$ being dense in $Y$.
\subsection*{Part b}
Assume $\phi$ has a polynomial left-inverse $\psi$. \\
Let $p+I(X)$ be an arbitrary element of the coordinate ring of $X$. \\
Note that $p \circ \psi \in k[y_1,...,y_m]$. \\
We have $\phi^* (p \circ \psi +I(Y)) =
p \circ \psi \circ \phi +I(x) =
p(\psi \circ \phi) +I(X) =p+I(X)$.\\ 
Thus $\phi^*$ is surjective. \\
Conversely assume $\phi^*$ is surjective. \\
Since $\phi^*$ is surjective, there exist $g_i \in k[y_1,...,y_m]$ such that 
$\phi (g_i + I(Y)) = x_i + I(X)$ for every $i \in \{1,2,...,n\}$. \\
So $(g_i \circ \phi) +I(X)=x_i +I(X) \rightarrow (g_i \circ \phi)(x)=x_i(x) \ \forall x \in X$.\\
Let $\psi = (g_1,g_2,...,g_n)$, then we clearly have
$\psi \circ \phi = id_{X}$.



\section*{Problem 4}
Since $k$ is a field, to prove bijectivity, it is sufficient to prove that $\phi$ is a ring homomorphism. \\
Let $a,b \in \mathbb{A}^1$. We have:
\begin{eqnarray*}
\phi(a+b)   &=&   (a+b)^p \\ &=&
a^p+{p \choose 1}a^{p-1}b+..
+{p \choose i}a^{p-1i}b^i+...+b^p
\\ &=&
a^p+b^p \ \ \ since  \ \ \ char(k)=p \ \ p \ \ is \ \ prime
\\ &=&
\phi(a)+\phi(b)
\end{eqnarray*}
Also $\phi(ab)=(ab)^p=a^pb^p=\phi(a)\phi(b)$.
Thus $\phi$ is a ring homomorphism.\\
Let $g \in k[x]$. Note that
$\phi^*(g)=g \circ \phi$. So $\phi^* :k[x] \rightarrow k[x^p]$. Thus $\phi^*$ is not an isomorphism as $k[x]$ is not isomorphic to $k[x^p]$ ($\phi^*$ is not even surjective). Hence $\phi$ is not an isomorphism.


\clearpage
\section*{Problem 5}
$I(y^2-x^2(x+1))=<y^2-x^2(x+1)>$ since $y^2-x^2(x+1)$ is irreducible.\\ 
So we have
$\bar{y}^2=\bar{x}^2(\bar{x}+\bar{1})$ in $\Gamma (V(y^2-x^2(x+1)))$. Thus\\
$$z=\frac{\bar{y}}{\bar{x}}=
\frac{\bar{x}^2(\bar{x}+\bar{1})}{\bar{x}\bar{y}}
=\frac{\bar{x}(\bar{x}+\bar{1})}{\bar{y}}$$
and
$$z^2=\frac{\bar{y}^2}{\bar{x}^2}
= \frac{\bar{x}^2(\bar{x}+\bar{1})}{\bar{x}^2}
=\bar{x}+\bar{1}$$
So $z^2$ has no poles since it has a polynomial representation and also $z^2 \in \Gamma(X)$.\\
We can also see that $z$ has no pole at $(a,b)$ if $a \not=0$ or $b \not=0$. So the only possible pole for $z$ is $(0,0)$. \\
Assume for a contradiction that $z$ is defined on $(0,0)$. So there exist $g,h \in \Gamma(X)$ such that $z=\frac{g}{h}$ and $h(0,0) \not=0$. Equivalently $h\bar{y}=g\bar{x}$. \\
Because of the relation $\bar{y}^2=\bar{x}^2(\bar{x}+\bar{1})$ any element
of $\Gamma(X)$ can be written uniquely in the form
$a(\bar{x})+b(\bar{x})\bar{y}$.\\
So we can write $h\bar{y}=g\bar{x}$ as:
$$(h_1(\bar{x})+h_2(\bar{x})\bar{y})\bar{y}=
(g_1(\bar{x})+g_2(\bar{x})\bar{y})\bar{x}$$
where $h_1(0) \not=0$. So
$$
h_1(\bar{x})\bar{y}+h_2(\bar{x})\bar{x}^2(\bar{x}+\bar{1})=
g_1(\bar{x})\bar{x}+g_2(\bar{x})\bar{x}\bar{y}
$$
Now by uniqueness we have $h_1(\bar{x})=g_2(\bar{x})\bar{x}$. Thus $h_1(0)=0$ which is a contradiction. \\
Hence $(0,0)$ is the only pole of $z$ and $z \not\in \Gamma(X)$.

\clearpage
\section*{Problem 6}
\subsection*{Part a}
Let $F(x,y)=ax^2+by^2+cxy+dx+ey+f \in k[x,y]$ be irreducible. We break the problem down into a few cases and subcases:
\begin{enumerate}
\item[Case 1:] Either $a$ or $b$ is nonzero. WLOG assume $a \not= 0$. \\
Let $X_1=\sqrt{a}(x+\frac{c}{2a}y)$. There exists $b_1$ such that $F=X_1^2+b_1y^2+dx+ey+f$. \\
There exist constants $d_1,e_1,f_1$ such that
$F=X_1^2+b_1y^2+d_1X_1+e_1y+f_1$. (It's very tedious to calculate these constants but they clearly exist).
\\
Let $X_2=X_1+\frac{d_1}{2}$. Then there exist constant $f_2$ such that
$F=X_2^2+b_1y^2+e_1y+f_2$. \\
We have 2 subcases to consider here:
\begin{enumerate}
\item[Subcase 1:] $b_1=0$. \\
So $F=X_2^2+e_1y+f_2$. Note that if $e_1=0$ we get
$F=(X_2-\sqrt{f_2})(X+\sqrt{f_2})$ which is reducible so $e_1 \not=0$. \\
Now let $Y=-e_1y-f_2$ and we have
$F=X_2^2-Y$.
\item[Subcase 2:] $b_1 \not=0$. \\
So $F=X_2^2+b_1y^2+e_1y+f_2=X_2^2+y^2+\frac{e_1}{b_1}y+\frac{f_2}{b_1}$. \\
Let $Y_1=y+\frac{e_1}{2b_1}$. Then there exist constant $f_3$ such that
$F=X_2^2+Y_1^2-f_3$. Note that if $f_3=0$ then
$F=(X_2-iY_1)(X_2+iY_1)$ which is reducible. So $f_3 \not=0$. \\
Now let $X_3=\sqrt{f_3}X_2$ and $Y_2=\sqrt{f_3}Y_1$. Then $F=f_3(X_3^2+Y_2^2-1)$. Therefore $V(F)$ can be written in the form $X_3^2+Y_2^2-1=0$.
\end{enumerate}
\item[Case 2:] $a=b=0$. \\
So $F=cxy+dx+ey+f$. Note that $c \not=0$ because otherwise thepolynomial would have degree $1$.
We have: \\
$F=c(xy+\frac{d}{c}x+\frac{e}{c}y)+f$. So there exist constant $c_1$ such that
$F=c(x+\frac{e}{c})(y+\frac{d}{c})+f_1$. Where $f_1 \not=0$ otherwise the polynomial would be reducible. \\

Let $X=\frac{-\sqrt{c}}{f_1}(\frac{1}{2}x+\frac{1}{2}y+\frac{e}{2c}+\frac{d}{2c})$ and 
$Y=\frac{-i \sqrt{c}}{f_1}(\frac{1}{2}x-\frac{1}{2}y+\frac{e}{2c}-\frac{d}{2c})$.\\ 
Then $X^2+Y^2-1=(X-iY)(X+iY)-1=\frac{-c}{f_1}(x+\frac{e}{c})(y+\frac{d}{c})-1$. \\
So $V(F)$  can be written in the form $X^2+Y^2-1=0$.

\clearpage
Now we consider the case that $char(k)=2$.\\
Let $F(x,y)=ax^2+by^2+cxy+dx+ey+f \in k[x,y]$ be irreducible. \\
\begin{enumerate}
\item[Case 1:] $c=0$ \\
Let $X=\sqrt{a}x+\sqrt{b}y$ and let $Y=-dx-ey-f$. Then $F=X^2-Y$. Thus $V(F)$ is isomorphic to $V(Y-X^2)$.
\item[Case 2:] $c \not=0$\\
We have 2 subcases
\begin{enumerate}
\item[Subcase 1:] $a=b=0$ \\
Let $X=cx+e$ and let $Y=y+\frac{d}{c}$. Then
$F=XY-f_1$ for some constant $f_1$. Thus $V(F)$ is isomorphic to $V(XY-1)$ and I will prove in part (b) of this problem that $V(XY-1)$ is isomorphic to $V(X^2+Y^2-1)$.
\item[Subcase 2:] Either $a$ or $b$ is nonzero. WLOG assume $a \not= 0$. \\
\end{enumerate}
\end{enumerate}


\clearpage
\subsection*{Part b}
The polynomial map $\phi: \mathbb{A}^1 \rightarrow V(y-x^2)$
which sends $t$ to $(t^2,t)$ is clearly an isomorphism because it's inverse simply sends $(t^2,t)$ to $t$ which is polynomial. So $V(y-x^2)$ is isomorphic to $\mathbb{A}^1$. \\
Claim: $V(x^2+y^2-1)$ is isomorphic to $V(xy-1)$.
\\
Proof: Let $\phi:V(x^2+y^2-1) \rightarrow V(xy-1)$ be a polynomial map that sends $(a,b)$ to $(a-ib,a+ib)$.\\ 
It's inverse $\phi^{-1}: V(xy-1)
\rightarrow V(x^2+y^2-1)$ sends $(a,b)$ to
$(\frac{1}{2}(a+b),i\frac{1}{2}(a-b))$. \\
Hence $V(x^2+y^2-1)$ is isomorphic to $V(xy-1)$.
\\
We proved in class that $V(xy-1)$ is not isomorphic to the affine line, thus $V(x^2+y^2-1)$ is not isomorphic to $\mathbb{A}^1$.

\subsection*{Part c}
Given a point $(a,b) \in V(x^2+y^2-1)$, we need to verify that the intersection of $y=0$ and the line containing $(a,b)$ and $(0,1)$ is the point $(\frac{a}{1-b},0)$. \\
The line passing through  $(a,b)$ and $(0,1)$
is $y-1=\frac{b-1}{a}x$. $y=0$, so
$-1=\frac{b-1}{a}x$. Thus $x=\frac{a}{1-b}$. \\
Hence the rational map sending $(x,y)$ to
$\frac{x}{1-y}$ is stereographic projection. \\
The inverse $\Theta^{-1} : \mathbb{A}^1 \rightarrow V(x^2+y^2-1)$ is also a rational map
that sends $t$ to $(\frac{2t}{1+t^2},\frac{1-t^2}{1+t^2})$. \\
Also note that both $\Theta^{-1}$ and $\Theta$ are dominant because\\ 
$cl(\Theta^{-1}(\mathbb{A}^1))=cl(X\setminus \{(0,1)\})=X$ and $cl(\Theta(X))=\mathbb{A}^1$.\\
Thus $\Theta$ is a birational equivalence.


\end{enumerate}








\end{document}