\documentclass[12pt]{article}

\usepackage{fullpage,url,amssymb,epsfig,color,xspace,enumerate}
\usepackage{amsmath}

\renewcommand{\thesubsection}{Problem \arabic{subsection}}

\begin{document}
\begin{center}
  {\Large\bf University of Waterloo}\\
  \vspace{3mm}
         {\Large\bf Pmath 450 - Summer 2015}\\
         \vspace{2mm}
                {\Large\bf Assignment 1}\\
                \vspace{3mm}
                \textbf{Sina Motevalli 20455091}
\end{center}
\section*{Problem 1}
\subsection*{Part a}


\subsection*{Part b}
The closed unit ball clearly contains the points of the form
$x_n=(0,0,...,1,0,0,0,...)$ (1 is on the $n$th position and everything else is 0). 
So the distance between any two of these points is $2^{\frac{1}{p}}$. Thus this sequence has no converging subsequesnce (because no subsequence is couchy). Hence the closed unit ball is not compact in $l^p$.
\subsection*{Part c}
Let $S=\{(q_n)_{n=1}^{\infty} : q_n \in \mathbb{Q} \ \
and \ \ \exists N \in \mathbb{N} \ \ with \ \
q_n=0, \forall n \ge N \ \ and \ \
||(q_n)||_p < \infty\}$. \\
Claim: $S$ is dense in $l^p$.\\
Proof: Let $(x_n)_{n=1}^{\infty} \in l^p$. 
Let $\epsilon > 0$. Let $N \in \mathbb{N}$ such that
$\sum_{n=N}^{\infty} |x_n|^p < \frac{\epsilon}{2}$.
\\
For each $i \in \{1,2,...,N-1\}$ find $q_i \in \mathbb{Q}$ such that $|q_i-x_i|^p < \frac{\epsilon}{2(N-1)}$. \\
Now consider the sequence
$(q_1,q_2,...,q_{N-1},0,0,0,0,.....)$. Now we compute the difference between the two sequences in $l^p$:
\begin{eqnarray*}
(||(q_n)-(x_n)||_p)^p &=& 
\sum_{i=1}^{N-1} |q_i-x_i|^p + \sum_{i=N}^{\infty}
|q_i-x_i|^p
\\ &=&
\sum_{i=1}^{N-1} |q_i-x_i|^p + \sum_{i=N}^{\infty}
|x_i|^p
\\ & < &
\frac{\epsilon}{2} + \frac{\epsilon}{2}
\\ &=&
\epsilon
\end{eqnarray*}
Thus $||(q_n)-(x_n)||_p < \epsilon^p$, but since $\epsilon$ was arbitrary and $p$ is a constant,$S$ is desne in $l^p$.

\clearpage
\section*{Problem 2}
\subsection*{Part a}
Let $A_n=[a,b-\frac{1}{n}]$. So We have:
$A_1 \subset A_2 \subset ... \subset \cup_{n=1}^{\infty}
A_n=[a,b)$. \\
By the continuity of measure we have:
\begin{eqnarray*}
m([a,b)) &=& \lim_{n \rightarrow \infty} m(A_n)
\\ &=&
\lim_{n \rightarrow \infty} m([a,b-\frac{1}{n}])
\\ &=&
\lim_{n \rightarrow \infty} b-\frac{1}{n}-a
\\ &=&
b-a
\end{eqnarray*}
\subsection*{Part b}
Let $A \subset \mathbb{R}$ be a lebesgue measurable set.
Let $t \in \mathbb{R}$. Need to show $A+t$ islebesgue measurabel. \\
Let $E \subset \mathbb{R}$. Since $A$ is measurable, we have
\begin{eqnarray}
m^*(E)=m^*(E \cap A)+m^*(E \cap A^c)
\end{eqnarray}
We have:
\begin{eqnarray*}
x \in E \cap A+t &\Longleftrightarrow &
x \in E \ \ and \ \ x \in A+t
\\ &\Longleftrightarrow &
x \in E \ \ and \ \ x-t \in A
\\ &\Longleftrightarrow &
x-t \in E-t \ \ and \ \ x-t \in A
\\ &\Longleftrightarrow &
x-t \in (E-t) \cap A
\\ &\Longleftrightarrow &
x \in (E-t \cap A) + t
\end{eqnarray*}
So $E \cap A+t = [(E-t) \cap A]+t$. Thus,
$$m^*(E \cap A+t)=m^*([(E-t) \cap A]+t)=m^*((E-t) \cap A)$$.
By a similar argument we get
$E \cap (A+t)^c = [(E-t) \cap A^c]+t$. Thus, 
$$m^*(E \cap (A+t)^c)=m^*([(E-t) \cap A^c]+t)=m^*((E-t) \cap A^c)$$.
So we have:
\begin{eqnarray*}
m^*(E \cap (A+t)^c)+m^*(E \cap A+t) &=&
m^*((E-t) \cap A^c)+m^*((E-t) \cap A) \\ &=&
m^*(E-t) \\ &=&
m^*(E)
\end{eqnarray*}
Hence $A+t$ is lebesgue measurable.

\clearpage
\section*{Problem 3}
We have 2 cases:\\
Case 1: $E$ is bounded. \\
Let $\epsilon >0$. Choose an open set $U$ such that $\bar{E} \cap E^c \subset U$ and $m(U) < m(\bar{E} \cap E^c)+\epsilon$. \\
Let $K=E \cap U^c = \bar{E} \cap U^c$. Since $K$ is closed and bounded, it is compact. We have:
\begin{eqnarray*}
m(E) = &=& m(\bar{E}) - m(\bar{E} \cap E^c)
\\ &<&
m(\bar{E})-m(U)+\epsilon
\\ &\le &
m(\bar{E} \cap U^c)+\epsilon
\\ &=&
m(K) + \epsilon
\end{eqnarray*}
Case 2: $E$ is unbounded. \\
Write $E=\cup E_j$ where each $E_j$ is bounded and $E_j \subset E_{j+1}$. Let $\epsilon > 0$. By the previous part, there exist a compact set $K_j \subset E_j$ for each $j$ such that $m(E_j) < m(K_j)+\epsilon$. Since $E_j \rightarrow E$, the result follows.
\section*{Problem 4}
Let $X \subset \mathbb{R}$ be open. Let $X \cap Q =\{p_1,p_2,...\}$.
For every $i \in \mathbb{N}$, we define
$$B_i = \bigcup_{p_i \in I \subset X}^{I  \ is  \ open \ interval} I$$
Claim:$X= \cup_{i=1}^{\infty} B_i$ \\
Proof: It is clear that $\cup_{i=1}^{\infty} B_i \subset X$ because every $B_i$ is inside $X$. So it remains to prove the other inclusion. 
Let $x \in X $. Since $X$ is open, there exist $\epsilon >0$ such that $(x-\epsilon,x+\epsilon) \subset X$. Since $\mathbb{Q}$ is dense in $\mathbb{R}$ we can pick a $p_j \in (x-\epsilon,x+\epsilon)$ where $p_j$ is a rational number. Then by definition of $B_j$ we have $p_j \in (x-\epsilon,x+\epsilon) \subset B_j \subset \cup_{i=1}^{\infty} B_i$. Thus $X \subset \cup_{i=1}^{\infty} B_i$ and we are done.\\
Note that $\cup_{i=1}^{\infty} B_i$ is a countable union of open intervals because each $B_i$ is a union of open intervals containing a common point, so it is an open interval.

\clearpage
\section*{Problem 5}
\subsection*{Part a}
Let $C_0=[0,1]$. We define $C_n$ recursively for $n \in \mathbb{N}$ as follows: \\
$C_n$ is defined as every interval of $C_{n-1}$ with the open middle third of each interval removed. \\
So $C_n$ contains $2^{n}$ intervals each of length $\frac{1}{3^n}$.
\\
Now we have:
$C=\cap_{n=0}^{\infty} C_n \subset ... \subset C_2 \subset C_1 \subset C_0$. \\
By downward continuity of measure, we have: \\
\begin{eqnarray*}
m(C) &=& \lim_{n \rightarrow \infty} m(C_n)
\\ &=&
\lim_{n \rightarrow \infty} \frac{2^{n}}{3^n}
\\ &=&
0
\end{eqnarray*}
Thus the lebesgue measure of Cantor set is zero.
\subsection*{Part b}
Since measure of the cantor set is $0$, every subset of cantor set is measurable with measure $0$, and since the cardinality of the set of subsets of cantor set is $2^{\mathbb{R}}$, the cardinality of the set of lebesgue measurable sets is also $2^{\mathbb{R}}$.

\clearpage
\subsection*{Part c}
Let $\alpha \in (0,1)$.\\
Let $(x_n)_{n=0}^{\infty} \in [0,1]$ be a decreasing sequence such that $\sum_{n=0}^{\infty} 2^nx_n \rightarrow 1-\alpha$. \\
Let $C_0=[0,1]$. We define $C_n$ recursively for $n \in \mathbb{N}$ as follows: \\
$C_n$ is defined as every interval of $C_{n-1}$ with the  middle open interval of length $x_n$ of each interval removed. \\
Then we define $C=\cap_{n=0}^{\infty} C_n$. First I should prove that this construction is possible.
Note that to construct the set $C_n$ we are making $2^n$ intervals each of length $\frac{1-\sum_{k=0}^{n-1} 2^kx_k}{2^n}$. So it suffices to prove that
$x_n < \frac{1-\sum_{k=0}^{n-1} 2^kx_k}{2^n}$.
Assume for a contradiction that
$x_n \ge \frac{1-\sum_{k=0}^{n-1} 2^kx_k}{2^n}$, we have:
\begin{eqnarray*}
x_n \ge \frac{1-\sum_{k=0}^{n-1} 2^kx_k}{2^n}
&\Rightarrow & 2^nx^n \ge 1-\sum_{k=0}^{n-1} 2^kx_k
\\ &\Rightarrow & 
\sum_{k=0}^{n} 2^kx_k \ge 1
\end{eqnarray*}
But this is a contradiction because $\sum_{n=1}^{\infty} 2^nx_n \rightarrow 1-\alpha < 1$. \\
Now I claim that $m(C)=\alpha$ because:
\begin{eqnarray*}
m(C) &=& m([0,1])-m(C^c \cap [0,1])
\\ &=&
1-L(intervals \ removed \ in \ the \ construction \ of \ C)
\\ &=&
1-\sum_{n=0}^{\infty} 2^nx_n
\\ &=&
1-(1-\alpha)
\\ &=&
\alpha
\end{eqnarray*}


\end{document}