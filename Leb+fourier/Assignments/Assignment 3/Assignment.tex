\documentclass[12pt]{article}

\usepackage{fullpage,url,amssymb,epsfig,color,xspace,enumerate}
\usepackage{amsmath}

\renewcommand{\thesubsection}{Problem \arabic{subsection}}

\begin{document}
\begin{center}
  {\Large\bf University of Waterloo}\\
  \vspace{3mm}
         {\Large\bf Pmath 450 - Summer 2015}\\
         \vspace{2mm}
                {\Large\bf Assignment 3}\\
                \vspace{3mm}
                \textbf{Sina Motevalli 20455091}
\end{center}
\section*{Problem 1}
Assume $f:\mathbb{R} \rightarrow \mathbb{R}$ is integrable. Then since $f^+$ and $f^-$ are non-negative and measurable, so we have:
\begin{eqnarray*}
\int_{\mathbb{R}} f(x+y)dm(x) &=&
\int_{\mathbb{R}} f^+(x+y)dm(x) - 
\int_{\mathbb{R}} f^-(x+y)dm(x)
\\ &=&
\int_{\mathbb{R}} f^+(x)dm(x) - 
\int_{\mathbb{R}} f^-(x)dm(x)
\\ &=&
\int_{\mathbb{R}} f(x)dm(x)
\end{eqnarray*}
Now if $f: \mathbb{R} \rightarrow \mathbb{C}$
is integrable, we have that $Re(f)$ and $Im(f)$ are integrable and therefore we have:
\begin{eqnarray*}
\int_{\mathbb{R}} f(x+y)dm(x) &=&
\int_{\mathbb{R}} Re(f(x+y))dm(x) +i 
\int_{\mathbb{R}} Im(f(x+y))dm(x)
\\ &=&
\int_{\mathbb{R}} Re(f(x))dm(x) - 
\int_{\mathbb{R}} Im(f(x))dm(x)
\\ &=&
\int_{\mathbb{R}} f(x)dm(x)
\end{eqnarray*}

\clearpage
\section*{Problem 2}
\subsection*{Part a}
We know that
$\sup \{|f(x)+g(x)|: x \in A\} \le 
\sup \{|f(x)|: x \in A\}+\sup \{|g(x)|: x \in A\}$. This implies that
$$\inf \{\sup |f(x)+g(x)|: x \in A\} 
\le 
\inf \{\sup |f(X)|:x \in A\} +
\inf \{\sup |g(x)|: x \in A\}$$
Hence $||f+g||_{\infty} \le ||f||_{\infty}+||g||_{\infty}$.
\subsection*{Part b}
We know that $m\{x : |h(x)| > ||h||_{\infty}\}=0$. Thus
$$\inf \{\alpha \in \mathbb{R}: m\{x:|h(x)| >\alpha \} =0\} \le ||h||_{\infty}$$
Now assume for a contradiction that there exist $\alpha < ||h||_{\infty}$ such that
$m\{x : |h(x)| > \alpha \} = 0$.\\
Since $||h||_{\infty}=\inf_{m(E\setminus A)=0}
\{\sup |h(x)|:x \in A\}$ and $\alpha < ||h||_{\infty}$, for every set $A$ with \\ 
$m(E\setminus A)=0$, we have that $\alpha < \{\sup |h(x)|: x \in A\}$.
This means that there exist no set $A$ with $m(E\setminus A)=0$ such that
$A=\{x: |h(x)| \le \alpha \}$. Thus $m\{x: |h(x)| > \alpha \} > 0$. Hence 
$$\inf \{\alpha \in \mathbb{R}: m\{x:|h(x)| >\alpha \} =0\} = ||h||_{\infty}$$


\section*{Problem 3}
By lotus lemma we have:
$$\int \lim \inf  f_n \le \lim \inf \int f_n$$
Since $f_n \rightarrow f$, we have 
$\int f \le \lim \inf \int f_n$. Thus $f$ is integrable. Now dominated convergence theorem readily implies $\int f = \lim_n \int f_n$.


\clearpage
\section*{Problem 4}
\subsection*{Part a}
Let $f(x)=0$ when $x<1$ and $f(x)=\frac{1}{x}$ when $x \ge 1$. We have:
$$\int_{\mathbb{R}} |f|^2 =\int_1^{\infty} \frac{1}{x^2} = 1$$
Thus $f \in  \ L^2(\mathbb{R})$. But
$\int_{\mathbb{R}} |f| =\int_1^{\infty} \frac{1}{x} $ does not converge, thus $f \not\in \ L^1(\mathbb{R})$. \\
Let $g(x)=\frac{1}{\sqrt{x}}$ on $[0,1]$ and $g(x)=0$ elsewhere. We have:
$$||g||_1=\int_{\mathbb{R}} g
=\int_0^1 \frac{1}{\sqrt{x}}=2$$
But $\int_{\mathbb{R}} g^2=\int_0^1 \frac{1}{x}=\infty$. Thus $g \not\in L^2(\mathbb{R})$.


\subsection*{Part b}
Let $f^2 \in L^1[0,1]$. So
$\int_0^1 |f^2| =\int_0^1 |f|^2 < \infty$, thus
$f \in L^2[0,1]$. We have:
\begin{eqnarray*}
\int_0^1 |f| &=& \int_0^1 |f|.1
\\ & \le &
||f||_2||1||_2 \ \ by \ \ holder's \ \ inequality
\end{eqnarray*} 
Since $f \in L^2[0,1]$, $||f||_2 < \infty$, so
$||f||_2||1||_2 < \infty$ which implies $\int_0^1 |f| < \infty$. \\
Hence $f \in L^1[0,1]$.

\clearpage
\section*{Problem 5}
Let $||f||_{\infty} > \epsilon >0$. Let $A_{\epsilon}=\{x: |f(x)| \ge ||f||_{\infty} - \epsilon\}$. So by definition of maximum norm we get that $m(A_{\epsilon}) >0$, so we have: 
$$||f||_p \ge \left(\int_{A_{\epsilon}} 
(||f||_{\infty}-\epsilon)^p \right)^{\frac{1}{p}}
=
(||f||_{\infty}-\epsilon)(m(A_{\epsilon}))^{\frac{1}{p}}
\rightarrow ||f||_{\infty}-\epsilon
\ \ as \ \ p \rightarrow \infty
$$
Thus, $\lim_{p \rightarrow \infty} \inf ||f||_p \ge ||f||_{\infty}$. \\
We also have:
\begin{eqnarray*}
||f||_p &=& 
\left( \int |f|^{p-1}|f| \right)^{\frac{1}{p}}
\\ &\le &
||f||_{\infty}^{\frac{p-1}{p}} ||f||_1^{\frac{1}{p}} \ \ \ by \ \ \ holder's \ \ \ inequality
\\ &\rightarrow &
||f||_{\infty} \ \ \ as \ p \rightarrow \infty
\end{eqnarray*}
Thus, $\lim_{p \rightarrow \infty} \sup ||f||_p \le ||f||_{\infty}$. \\
Hence $||f||_p \rightarrow ||f||_{\infty}$ as $p \rightarrow \infty$.



\clearpage
\section*{Problem 6}
Claim: $S$ is dense in $C[0,1]$ with respect to $L^2[0,1]$ norm. \\
Proof: \\
Let $\epsilon >0$. Let $f \in C[0,1]$. \\ 
WLOG we can assume that $f$ is real-valued. (Otherwise approximate real and imaginary part and put them back together).\\
First assume that $f$ is bounded. Say $|f(x)| \le N \ \forall x \in [0,1]$.\\
We define a new function $g: [0,1] \rightarrow \mathbb{R}$ as follows: \\\
$g(x)=f(x)$ for all $x \in (\epsilon,1-\epsilon)$.
\\
On $[0,\epsilon]$, $g$ is the line from $0$ to $f(\epsilon)$ ($g(0)=0$ and $g(\epsilon)=f(\epsilon)$). 
\\
On $[1-\epsilon,1]$, $g$ is the line from $f(1-\epsilon)$ to $0$ ($g(1-\epsilon)=f(1-\epsilon)$ and $g(1)=0$). \\
Note that $g \in S$. We have:
\begin{eqnarray*}
||f-g||_2^2 &=&
\int_0^1 |f-g|^2
\\ &=&
\int_{[0,\epsilon]} |f-g|^2 +
\int_{(\epsilon,1-\epsilon)} |f-g|^2 +
\int_{[1-\epsilon,1]} |f-g|^2
\\ &\le &
N^2\epsilon + 0 + N^2\epsilon
\\ &=&
2N^2\epsilon 
\end{eqnarray*}
This concludes the proof for $f$ being bounded. \\
Now suppose $f \in C[0,1]$ is arbitrary. \\
Define $f_N(x)=f(x)$ if $|f(x)| \le N$ and
$f_N(x)=0$ otherwise.\\
We have $f_N \rightarrow f$ pointwise a.e. \\
So $|f_N-f|^2 \rightarrow 0$ pointwise a.e.\\
Since $|f-f_N|^2 \le |f|^2$ and $|f|^2$ is integrable, by dominated convergence theorem, we have:
$$\int_{[0,1]} |f-f_N|^2 \rightarrow \int_{[0,1]} 0=0$$
So $||f-F_N||_2 \rightarrow 0$. \\
Let $\epsilon >0$. Pick $N \in \mathbb{N}$ such that
$||f-f_N||_2 < \frac{\epsilon}{2}$. \\
Get $h \in S$ with
$||h-f_N||_2 < \frac{\epsilon}{2}$. We have:
$$||h-f||_2 \le ||h-f_N||_2+||f_N-f||_2 < \epsilon$$
Hence $S$ is dense in $C[0,1]$ with respect to $L^2[0,1]$ norm. \\
Let $(g_n)_{n=1}^{\infty}$ be a sequence in $S$ such that $||g_n||_2 \rightarrow ||f||_2$ and $g_n \le f$ for all $n$. \\
By the dominated convergence theorem,
$\int_0^1 f^2=\int_0^1 \lim_{n \rightarrow \infty} fg_n
=\lim_{n \rightarrow \infty} \int_0^1 fg_n=0$ since $\int_0^1 fg=0$ for all $g \in S$. \\
Thus $||f||_2 =0$. Hence $f=0$ a.e.

\clearpage
\subsection*{Problem 7}
Since $f \ge 0$, $||f^n||_1=\int_0^1 f^n(x)=\int_0^1 f(x)=||f||_1$ for all $n \in \mathbb{N}$.
Now since $||f^n||_1=||f||_1$ we have that
$f^n(x)=f(x)$ a.e for all $n \in \mathbb{N}$. \\
So $f(x)=1$ a.e. \\
Let $E=\{x:f(x)=1\}$. We just need to prove that $E$ is measurable. We have: \\
\begin{eqnarray*}
E &=& \left(
\bigcap_{n=1}^{\infty} \{x:f(x) \le 1+\frac{1}{n}\} 
\right)
\cap
\left(
\bigcap_{n=1}^{\infty} \{x:f(x) \ge 1-\frac{1}{n}\}
\right)
\end{eqnarray*} 
Since countable intersection of measurable sets is measurable, $E$ is measurable and $f=X_E$ a.e.




\end{document}